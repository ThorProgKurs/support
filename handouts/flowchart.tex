\documentclass[page,number,smaller]{beamer}
% \documentclass[a4paper]{article}

\usepackage[UKenglish]{babel}
%Mathekram:
\usepackage{amsmath,amssymb,mathrsfs}
\usepackage{enumerate}
\usepackage{makeidx}
\usepackage{times}
\usepackage{multimedia}

\usepackage{tikz}


\begin{document}



\tikzstyle{box} = [rectangle, rounded corners, minimum width=3cm, minimum height=1cm,text centered, draw=black, fill=blue!10]
\tikzstyle{arrow} = [line width=2pt,->,>=stealth]

\vphantom{TEXT}

\resizebox{\textwidth}{!}{
\begin{tikzpicture}[node distance=3cm]

\node (plan) [box] {{\Large{ \textbf{Programm planen}}} };

\node (code) [box, below of=plan] {\Large \textbf{Coden} };

\node (compile) [box, below of=code] {\Large \textbf{Kompilieren} };

\node (read) [box, left of=compile, xshift=-6cm, align=left] {\Large \textbf{Fehlermeldungen lesen} \\ \textbf{\Large und verstehen} };

\node (test) [box, below of=compile, yshift=-1cm] {\Large \textbf{Programm testen}};

\node (finish) [box, below of=test] {\Huge\textbf{ \reflectbox{/}o/}};

\node (debug) [box, right of=test, xshift=5cm] {\Large \textbf{Debuggen}};

\draw [arrow, color=green] (plan) -- (code);

\draw [arrow, color=green] (code) -- (compile);

\draw [arrow, color=green] (compile) -- node[anchor=west, align=left, color=black] {Es wird ohne \\ Fehler kompiliert} (test);

\draw [arrow, color=red] (compile) -- node[anchor=north, align=center, color=black] {Es gibt Fehler\\oder Warnungen}(read);

\draw [arrow] (read) |- node[anchor=north, align=left, xshift=3cm] {Fehler und Warnungen beheben}(code);

\draw [arrow, color=green] (test) -- node[anchor=east, align=right, color=black] {Programm tut\\ was es soll} (finish);

\draw [arrow] (debug) |-node[anchor=north, align=right, xshift=-2cm] {Die Fehler liegen\\im Code} (code);

\draw [arrow] (debug) |- node[anchor=north, align=right, xshift=-2cm] {Die Fehler liegen\\im Algorithmus} (plan);

\draw [arrow, color=red] (test) --node[anchor=north, align=center, color=black] {Programm tut\\nicht was es soll} (debug);

\end{tikzpicture}
}
\end{document}
